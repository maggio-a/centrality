% !TeX root = p2p.tex

\section{Experiments}
\label{sec:exp}

The aim of the experiments was to compare the number of messages required by the algorithms to complete the decentralized computation of the indices, and to measure the error of the estimates of \multibfs{} with different values of the parameter $p$.

The results were obtained by running simulations on the following networks (taken from the KONECT \cite{konect} repository):
\begin{description}[leftmargin=0cm]
 \item[\texttt{dolphins}] A social network of bottlenose dolphins where nodes are dolphins and edges represent frequent associations between individuals. ($n=62$, $m=159$, source \cite{network:dolphins}).
 \item[\texttt{surf}] A network that represents interpersonal contacts between windsurfers in southern California during the fall of 1986. ($n=62$, $m=336$, source \cite{network:surf}).
 \item[\texttt{macaques}] The undirected version of a directed network that represent dominance behavior in a colony of 62 adult female Japanese macaques -- obtained by forcing the edges to be symmetric. ($n=62$, $m=1167$, original source \cite{network:mac}).
 \item[\texttt{train}] A network that contains contacts between suspected terrorists involved in the train bombing of Madrid on March 11, 2004 as reconstructed from newspapers. ($n=62$, $m=243$, source \cite{network:train}).
 \item[\texttt{email}] The email communication network at the University Rovira i Virgili in Tarragona in the south of Catalonia in Spain. ($n=1133$, $m=5451$, source \cite{network:email}).
 \item[\texttt{powergrid}] The high-voltage power grid in the Western States of the United States of America. ($n=4941$, $m=6594$, source \cite{network:powergrid}).
\end{description}

\subsection{Performance evaluation}

The performance measures used to evaluate the algorithms are the number of steps required to complete the computation and the number of messages that the network nodes generate.

Table \ref{table:comparison} reports the results obtained by running \deccen{} and \multibfs{} with $p = 1$ (in order to compute exact values). Due to the high memory requirements of \deccen{} the simulations could only be performed on small networks, but the algorithms can be expected to behave similarly in any other scenario since all the messages are generated in a deterministic way.

\begin{table}
\centering

\begin{tabular}{r c c c c c c c}
\toprule
 & & & & \multicolumn{2}{c}{\deccen{}} & \multicolumn{2}{c}{\multibfs{}} \\ \cmidrule(lr){5-6} \cmidrule(lr){7-8} 

Network           & $n$ & $m$  & $\Delta$ & Steps  & Messages & Steps & Messages \\ \midrule

\texttt{surf}     & 43  & 336  & 3        & 6      & 181374  & 7     & 28896 \\
\texttt{dolphins} & 62  & 159  & 8        & 16     & 140051   & 17    & 19716 \\
\texttt{macaques}  & 62  & 1167 & 2        & 4      & 1458368  & 5     & 144708 \\
\texttt{train}    & 64  & 243  & 6        & 12     & 272673  & 13    & 31104 \\
 
\bottomrule

\end{tabular}

\caption{Steps required to complete and number of exchanged messages by \deccen{} and \multibfs{} executed with $p=1$.}

\label{table:comparison}

\end{table}

As expected, both algorithms require a number of steps to complete the execution that is twice the diameter $\Delta$ of the network, since each of the two ``discovery'' phases takes at most $\Delta$ steps to reach any destination from any given source, and the same also holds for the report phase of \deccen{} or the backtracking in \multibfs{}. In the test instances \multibfs{} exhibits an extra delay of one step between reaching the farthest child nodes and the beginning of the backtracking phase: this is bound to happen whenever nodes at distance $\Delta$ from a source have a non-empty siblings set, as the siblings need to be identified before reporting back.

The experimental results also show that \deccen{} requires a significantly larger amount of messages than \multibfs{} to terminate the execution, as anticipated in the previous sections. This is a consequence of the different way in which the algorithms handle the report phase: while in \deccen{} new reports are broadcast at every step of a discovery, \multibfs{} uses the distance from the source of the discovery to rank the nodes, and only generates reports from the most distant nodes (that are then routed back to the source using the predecessor sets).

\subsection{\multibfs{} approximations}

The quality of the approximations obtained with \multibfs{} were evaluated by running simulations with increasing values of $p$ (from 0.05 to 1.0 at steps of 0.05). The goal was to measure the numerical error of the estimates generated by the algorithm, and to evaluate how close the ``ranking'' of the nodes using these estimates is to the actual ranking of the nodes according to the exact centrality values, since nodes could be ranked correctly even if the estimates are subject to high error (this is especially interesting if a node wants to make a ``greedy'' choice knowing only its and it's neighbors index's value, as it gives an idea of how often this choice may be the wrong one). Each experiment was repeated ten times.

Figure \ref{fig:error} reports the average relative error $\epsilon_r$ yielded by the centrality estimators in the \texttt{dolphins}, \texttt{email} and \texttt{powergrid} networks. The data shows satisfactory results for the estimation of closeness centrality for small values of $p$, while stress and betweenness estimates exhibit large numerical error even for values of $p$ close to 1.

\begin{figure}
\centering
\begin{tikzpicture}
\begin{axis}
[
  height=0.28\textheight,
  width = 0.92\textwidth,
  xlabel={$p$},
  ylabel={Relative error $\epsilon_r$},
  xlabel near ticks, ylabel near ticks,
  label style = {
    font = \footnotesize
  },
  ticklabel style = {
    font = \footnotesize
  },
  legend style = {
    draw=none,
    font=\footnotesize
  },
  title = {Approximation of centralities in the \texttt{dolphins} network},
  title style = {
    font=\footnotesize
  }
]
\addplot+ [smooth, mark=none, semithick, red] table [x=Fraction, y=CCerr]{../results/dolphins/analysis_dolphins.txt}; \addlegendentry{$C_C$}
\addplot+ [smooth, mark=none, semithick, blue] table [x=Fraction, y=SCerr]{../results/dolphins/analysis_dolphins.txt}; \addlegendentry{$S_C$}
\addplot+ [smooth, mark=none, semithick, black] table [x=Fraction, y=BCerr]{../results/dolphins/analysis_dolphins.txt}; \addlegendentry{$B_C$}
\end{axis}
\end{tikzpicture}

\bigskip

\begin{tikzpicture}
\begin{axis}
[
  height=0.28\textheight,
  width = 0.92\textwidth,
  xlabel={$p$},
  ylabel={Relative error $\epsilon_r$},
  xlabel near ticks, ylabel near ticks,
  label style = {
    font = \footnotesize
  },
  ticklabel style = {
    font = \footnotesize
  },
  legend style = {
    draw=none,
    font=\footnotesize
  },
  title = {Approximation of centralities in the \texttt{email} network},
  title style = {
    font=\footnotesize
  }
]
\addplot+ [smooth, mark=none, semithick, red] table [x=Fraction, y=CCerr]{../results/arenas-email/analysis_arenas-email.txt}; \addlegendentry{$C_C$}
\addplot+ [smooth, mark=none, semithick, blue] table [x=Fraction, y=SCerr]{../results/arenas-email/analysis_arenas-email.txt}; \addlegendentry{$S_C$}
\addplot+ [smooth, mark=none, semithick, black] table [x=Fraction, y=BCerr]{../results/arenas-email/analysis_arenas-email.txt}; \addlegendentry{$B_C$}
\end{axis}
\end{tikzpicture}

\bigskip

\begin{tikzpicture}
\begin{axis}
[
  height=0.28\textheight,
  width = 0.92\textwidth,
  xlabel={$p$},
  ylabel={Relative error $\epsilon_r$},
  xlabel near ticks, ylabel near ticks,
  label style = {
    font = \footnotesize
  },
  ticklabel style = {
    font = \footnotesize
  },
  legend style = {
    draw=none,
    font=\footnotesize
  },
  title = {Approximation of centralities in the \texttt{powergrid} network},
  title style = {
    font=\footnotesize
  }
]
\addplot+ [smooth, mark=none, semithick, red] table [x=Fraction, y=CCerr]{../results/opsahl-powergrid/analysis_opsahl-powergrid.txt}; \addlegendentry{$C_C$}
\addplot+ [smooth, mark=none, semithick, blue] table [x=Fraction, y=SCerr]{../results/opsahl-powergrid/analysis_opsahl-powergrid.txt}; \addlegendentry{$S_C$}
\addplot+ [smooth, mark=none, semithick, black] table [x=Fraction, y=BCerr]{../results/opsahl-powergrid/analysis_opsahl-powergrid.txt}; \addlegendentry{$B_C$}
\end{axis}
\end{tikzpicture}
\caption{Approximation error in the estimation of centrality indices.}
\label{fig:error}
\end{figure}


The accuracy of the ranking induced by the approximated indices is measured by counting, among all the pairs of nodes, the fraction of pairs in which the nodes are wrongly ordered with respect to the ranking imposed by the exact centrality values; results are reported in figure \ref{fig:inversion}. In this case the results are encouraging: even for small values of $p$ the fraction of pairs in wrong order is relatively small.

\begin{figure}
\centering
\begin{tikzpicture}
\begin{axis}
[
  height=0.28\textheight,
  width = 0.92\textwidth,
  xlabel={$p$},
  ylabel={Fraction of pairs},
  xlabel near ticks, ylabel near ticks,
  label style = {
    font = \footnotesize
  },
  ticklabel style = {
    font = \footnotesize
  },
  cycle list name=color list,
  legend style = {
    draw=none,
    font=\footnotesize
  },
  legend plot pos = left,
  legend cell align = left,
  legend entries = {$C_C$, $S_C$, $B_C$},
  title = {Pairs in wrong rank order in the \texttt{dolphins} network},
  title style = {
    font=\footnotesize
  }
]
\addplot+ [smooth, semithick] table [x=Fraction, y=CCpercInv]{../results/dolphins/analysis_dolphins.txt};
\addplot+ [smooth, semithick] table [x=Fraction, y=SCpercInv]{../results/dolphins/analysis_dolphins.txt};
\addplot+ [smooth, semithick] table [x=Fraction, y=BCpercInv]{../results/dolphins/analysis_dolphins.txt};
\end{axis}
\end{tikzpicture}

\bigskip

\begin{tikzpicture}
\begin{axis}
[
  height=0.28\textheight,
  width = 0.92\textwidth,
  xlabel={$p$},
  ylabel={Fraction of pairs},
  xlabel near ticks, ylabel near ticks,
  label style = {
    font = \footnotesize
  },
  ticklabel style = {
    font = \footnotesize
  },
  yticklabel style = {/pgf/number format/fixed},
  scaled y ticks = false,
  cycle list name=color list,
  legend style = {
    draw=none,
    font=\footnotesize
  },
  legend plot pos = left,
  legend cell align = left,
  legend entries = {$C_C$, $S_C$, $B_C$},
  title = {Pairs in wrong rank order in the \texttt{email} network},
  title style = {
    font=\footnotesize
  }
]
\addplot+ [smooth, semithick] table [x=Fraction, y=CCpercInv]{../results/arenas-email/analysis_arenas-email.txt};
\addplot+ [smooth, semithick] table [x=Fraction, y=SCpercInv]{../results/arenas-email/analysis_arenas-email.txt};
\addplot+ [smooth, semithick] table [x=Fraction, y=BCpercInv]{../results/arenas-email/analysis_arenas-email.txt};
\end{axis}
\end{tikzpicture}

\bigskip

\begin{tikzpicture}
\begin{axis}
[
  height=0.28\textheight,
  width = 0.92\textwidth,
  xlabel={$p$},
  ylabel={Fraction of pairs},
  xlabel near ticks, ylabel near ticks,
  label style = {
    font = \footnotesize
  },
  ticklabel style = {
    font = \footnotesize
  },
  yticklabel style = {/pgf/number format/fixed},
  scaled y ticks = false,
  cycle list name=color list,
  legend style = {
    draw=none,
    font=\footnotesize
  },
  legend plot pos = left,
  legend cell align = left,
  legend entries = {$C_C$, $S_C$, $B_C$},
  title = {Pairs in wrong rank order in the \texttt{powergrid} network},
  title style = {
    font=\footnotesize
  }
]
\addplot+ [smooth, semithick] table [x=Fraction, y=CCpercInv]{../results/opsahl-powergrid/analysis_opsahl-powergrid.txt};
\addplot+ [smooth, semithick] table [x=Fraction, y=SCpercInv]{../results/opsahl-powergrid/analysis_opsahl-powergrid.txt};
\addplot+ [smooth, semithick] table [x=Fraction, y=BCpercInv]{../results/opsahl-powergrid/analysis_opsahl-powergrid.txt};
\end{axis}
\end{tikzpicture}
\caption{Pairs in wrong rank order with approximated centralities}
\label{fig:inversion}
\end{figure}

