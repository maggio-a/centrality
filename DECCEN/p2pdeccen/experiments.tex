% !TeX root = p2p.tex

\section{Experiments}

Simulations were performed to evaluate the performances of the algorithms, and to assess the quality of the estimations obtained by running \multibfs{} with different values of the parameter $p$.

The networks used in the experiments (taken from the KONECT repository \cite{konect}) are:
\begin{description}[style=standard]
 \item[\texttt{dolphins}] This is a social network of bottlenose dolphins. The nodes are the bottlenose dolphins (genus \emph{Tursiops}) of a bottlenose dolphin community living off Doubtful Sound, a fjord in New Zealand. An edge indicates a frequent association. The dolphins were observed between 1994 and 2001. ($n=62$, $m=159$, source \cite{network:dolphins}).
 \item[\texttt{surf}] This network contains interpersonal contacts between windsurfers in southern California during the fall of 1986. ($n=62$, $m=336$, source \cite{network:surf}).
 \item[\texttt{macaques}] This directed network contains dominance behaviour in a colony of 62 adult female Japanese macaques. An undirected version of the network (where edges are made symmetric) was used to run the experiments. ($n=62$, $m=1167$, source \cite{network:mac}).
 \item[\texttt{train}] This network contains contacts between suspected terrorists involved in the train bombing of Madrid on March 11, 2004 as reconstructed from newspapers. ($n=62$, $m=243$, source \cite{network:train}).
 \item[\texttt{email}] This is the email communication network at the University Rovira i Virgili in Tarragona in the south of Catalonia in Spain. Edges represent contacts between users. ($n=1133$, $m=5451$, source \cite{network:email}).
 \item[\texttt{powergrid}] This network is the high-voltage power grid in the Western States of the United States of America. The nodes are transformers, substations, and generators, and the ties are high-voltage transmission lines. ($n=4941$, $m=6594$, source \cite{network:powergrid}).
\end{description}

\subsection{Performance analysis}

The performance metrics used to evaluate the algorithms were the number of steps required to complete the computation and the number of messages that the agents needed to generate to do so. Note that due to the high memory requirements of \deccen{}, these simulations were performed on small networks.

\begin{table}
\centering

\begin{tabular}{r c c c c c c c}
\toprule
 & & & & \multicolumn{2}{c}{\deccen{}} & \multicolumn{2}{c}{\multibfs{}} \\ \cmidrule(lr){5-6} \cmidrule(lr){7-8} 

Network           & $n$ & $m$  & $\Delta$ & Steps  & Messages & Steps & Messages \\ \midrule

\texttt{surf}     & 43  & 336  & 3        & 6      & 1160310  & 6     & 28896 \\
\texttt{dolphins} & 62  & 159  & 8        & 16     & 985403   & 16    & 19716 \\
\texttt{macaques}  & 62  & 1167 & 2        & 4      & 8708620  & 4     & 144708 \\
\texttt{train}    & 64  & 243  & 6        & 12     & 1729397  & 12    & 31104 \\
 
\bottomrule

\end{tabular}

\caption{Steps required to complete and number of exchanged messages by \deccen{} and \multibfs{} executed with $p=1$.}

\label{table:comparison}

\end{table}

Table \ref{table:comparison} reports the results obtained by running \deccen{} and \multibfs{} with $p = 1$ in order to compute the exact centrality values. As expected, in both cases the number of steps required to complete is exactly twice the diameter $\Delta$ of the networks, since each ``discovery'' phase takes at most $\Delta$ steps to reach any destination from a given source, and the same also holds for the report phase of \deccen{} or the backtracking in \multibfs{}. However, the number of messages required by \deccen{} is significantly larger. This follows from the different way in which the report phase evolves in the two algorithms: while in \deccen{} any node must generate a report for each source and broadcast it to every other node in the network, in \multibfs{} reports are routed back to the source by signaling only the predecessors at each step.



%
%\begin{figure}
%
%\centering
%
%\begin{tikzpicture}
%\begin{axis}
%[
%  width=8cm,
%  height=\textheight*0.2,
%  xlabel={$p$},
%  ylabel={$\epsilon_r$},
%  xlabel near ticks, ylabel near ticks,
%  label style = {
%    font = \footnotesize
%  },
%  ticklabel style = {
%    font = \footnotesize
%  },
%  cycle list name=color list,
%  legend style = {
%    draw=none,
%    font=\footnotesize
%  },
%  legend plot pos = left,
%  legend cell align = left,
%  legend entries = {All nodes, 90th percentile},
%  title = {Closeness centrality approximation},
%  title style = {
%    font=\footnotesize
%  }
%]
%
%\addplot+ [smooth, semithick] table [x=Fraction, y=CCerr]{../results/dolphins/analysis_dolphins.txt};
%\addplot+ [smooth, mark=none, semithick] table [x=Fraction, y=CCPercentileErr]{../results/dolphins/analysis_dolphins.txt};
%
%\end{axis}
%\end{tikzpicture}
%
%\bigskip
%
%\begin{tikzpicture}
%\begin{axis}
%[
%  width=8cm,
%  height=\textheight*0.2,
%  xlabel={$p$},
%  ylabel={$\epsilon_r$},
%  xlabel near ticks, ylabel near ticks,
%  label style = {
%    font = \footnotesize
%  },
%  ticklabel style = {
%    font = \footnotesize
%  },
%  cycle list name=color list,
%  legend style = {
%    draw=none,
%    font=\footnotesize
%  },
%  legend plot pos = left,
%  legend cell align = left,
%  legend entries = {All nodes, 90th percentile},
%  title = {Stress centrality approximation},
%  title style = {
%    font=\footnotesize
%  }
%]
%
%\addplot+ [smooth, semithick] table [x=Fraction, y=SCerr]{../results/dolphins/analysis_dolphins.txt};
%\addplot+ [smooth, mark=none, semithick] table [x=Fraction, y=SCPercentileErr]{../results/dolphins/analysis_dolphins.txt};
%
%\end{axis}
%\end{tikzpicture}
%
%\bigskip
%
%\begin{tikzpicture}
%\begin{axis}
%[
%  width=8cm,
%  height=\textheight*0.2,
%  xlabel={$p$},
%  ylabel={$\epsilon_r$},
%  xlabel near ticks, ylabel near ticks,
%  label style = {
%    font = \footnotesize
%  },
%  ticklabel style = {
%    font = \footnotesize
%  },
%  cycle list name=color list,
%  legend style = {
%    draw=none,
%    font=\footnotesize
%  },
%  legend plot pos = left,
%  legend cell align = left,
%  legend entries = {All nodes, 90th percentile},
%  title = {Betweenness centrality approximation},
%  title style = {
%    font=\footnotesize
%  }
%]
%
%\addplot+ [smooth, semithick] table [x=Fraction, y=BCerr]{../results/dolphins/analysis_dolphins.txt};
%\addplot+ [smooth, mark=none, semithick] table [x=Fraction, y=BCPercentileErr]{../results/dolphins/analysis_dolphins.txt};
%
%\end{axis}
%\end{tikzpicture}
%
%\caption{Approximation of centrality indices in the \texttt{dolphins} network}
%
%\end{figure}



















\subsection{\multibfs{} approximations}
The quality of the estimations yielded by \multibfs{} was evaluated both in terms of the numerical error introduced by the estimators, and the difference in the ranking of the nodes induced by the centrality values which could prove to be accurate even in presence of non-negligible error in the estimates.

Figure \ref{fig:error} reports the average relative error $\epsilon_r$ yielded by the centrality estimators for increasing values of the parameter $p$ in the \texttt{dolphins}, \texttt{email} and \texttt{powergrid} networks. The estimation of Closeness centrality is much more accurate than the estimation of Stress and Betweenness centrality, which behave roughly the same.

\begin{figure}
\centering
\begin{tikzpicture}
\begin{axis}
[
  height=0.3\textheight,
  width = 0.92\textwidth,
  xlabel={$p$},
  ylabel={Relative error $\epsilon_r$},
  xlabel near ticks, ylabel near ticks,
  label style = {
    font = \footnotesize
  },
  ticklabel style = {
    font = \footnotesize
  },
  legend style = {
    draw=none,
    font=\footnotesize
  },
  title = {Approximation of centralities in the \texttt{dolphins} network},
  title style = {
    font=\footnotesize
  }
]
\addplot+ [smooth, mark=none, semithick, red] table [x=Fraction, y=CCerr]{../results/dolphins/analysis_dolphins.txt}; \addlegendentry{$C_C$}
\addplot+ [smooth, mark=none, semithick, blue] table [x=Fraction, y=SCerr]{../results/dolphins/analysis_dolphins.txt}; \addlegendentry{$S_C$}
\addplot+ [smooth, mark=none, semithick, black] table [x=Fraction, y=BCerr]{../results/dolphins/analysis_dolphins.txt}; \addlegendentry{$B_C$}
\end{axis}
\end{tikzpicture}

\bigskip

\begin{tikzpicture}
\begin{axis}
[
  height=0.3\textheight,
  width = 0.92\textwidth,
  xlabel={$p$},
  ylabel={Relative error $\epsilon_r$},
  xlabel near ticks, ylabel near ticks,
  label style = {
    font = \footnotesize
  },
  ticklabel style = {
    font = \footnotesize
  },
  legend style = {
    draw=none,
    font=\footnotesize
  },
  title = {Approximation of centralities in the \texttt{email} network},
  title style = {
    font=\footnotesize
  }
]
\addplot+ [smooth, mark=none, semithick, red] table [x=Fraction, y=CCerr]{../results/arenas-email/analysis_arenas-email.txt}; \addlegendentry{$C_C$}
\addplot+ [smooth, mark=none, semithick, blue] table [x=Fraction, y=SCerr]{../results/arenas-email/analysis_arenas-email.txt}; \addlegendentry{$S_C$}
\addplot+ [smooth, mark=none, semithick, black] table [x=Fraction, y=BCerr]{../results/arenas-email/analysis_arenas-email.txt}; \addlegendentry{$B_C$}
\end{axis}
\end{tikzpicture}

\bigskip

\begin{tikzpicture}
\begin{axis}
[
  height=0.3\textheight,
  width = 0.92\textwidth,
  xlabel={$p$},
  ylabel={Relative error $\epsilon_r$},
  xlabel near ticks, ylabel near ticks,
  label style = {
    font = \footnotesize
  },
  ticklabel style = {
    font = \footnotesize
  },
  legend style = {
    draw=none,
    font=\footnotesize
  },
  title = {Approximation of centralities in the \texttt{powergrid} network},
  title style = {
    font=\footnotesize
  }
]
\addplot+ [smooth, mark=none, semithick, red] table [x=Fraction, y=CCerr]{../results/opsahl-powergrid/analysis_opsahl-powergrid.txt}; \addlegendentry{$C_C$}
\addplot+ [smooth, mark=none, semithick, blue] table [x=Fraction, y=SCerr]{../results/opsahl-powergrid/analysis_opsahl-powergrid.txt}; \addlegendentry{$S_C$}
\addplot+ [smooth, mark=none, semithick, black] table [x=Fraction, y=BCerr]{../results/opsahl-powergrid/analysis_opsahl-powergrid.txt}; \addlegendentry{$B_C$}
\end{axis}
\end{tikzpicture}
\caption{Approximation error in the estimation of centrality indices.}
\label{fig:error}
\end{figure}


The accuracy of the ranking is measured by counting among all the pair of nodes, the fraction of pairs in which the nodes are wrongly ordered with respect to the ranking induced by the exact centrality values. Results are reported in figures \ref{fig:inversion:dolphins}--\ref{fig:inversion:powergrid}. In this case, even for small values of $p$ the fraction of pairs wrongly ranked is relatively small. This is encouraging if the indices are used locally, to make decisions based on the values computed at a node and at its neighbors.

\begin{figure}
\centering
\begin{tikzpicture}
\begin{axis}
[
  height=0.3\textheight,
  width = 0.92\textwidth,
  xlabel={$p$},
  ylabel={Fraction of pairs},
  xlabel near ticks, ylabel near ticks,
  label style = {
    font = \footnotesize
  },
  ticklabel style = {
    font = \footnotesize
  },
  cycle list name=color list,
  legend style = {
    draw=none,
    font=\footnotesize
  },
  legend plot pos = left,
  legend cell align = left,
  legend entries = {$C_C$, $S_C$, $B_C$},
  title = {Inversion},
  title style = {
    font=\footnotesize
  }
]
\addplot+ [smooth, semithick] table [x=Fraction, y=CCpercInv]{../results/dolphins/analysis_dolphins.txt};
\addplot+ [smooth, semithick] table [x=Fraction, y=SCpercInv]{../results/dolphins/analysis_dolphins.txt};
\addplot+ [smooth, semithick] table [x=Fraction, y=BCpercInv]{../results/dolphins/analysis_dolphins.txt};
\end{axis}
\end{tikzpicture}

\bigskip

\begin{tikzpicture}
\begin{axis}
[
  height=0.3\textheight,
  width = 0.92\textwidth,
  xlabel={$p$},
  ylabel={Fraction of pairs},
  xlabel near ticks, ylabel near ticks,
  label style = {
    font = \footnotesize
  },
  ticklabel style = {
    font = \footnotesize
  },
  yticklabel style = {/pgf/number format/fixed},
  scaled y ticks = false,
  cycle list name=color list,
  legend style = {
    draw=none,
    font=\footnotesize
  },
  legend plot pos = left,
  legend cell align = left,
  legend entries = {$C_C$, $S_C$, $B_C$},
  title = {Inversion},
  title style = {
    font=\footnotesize
  }
]
\addplot+ [smooth, semithick] table [x=Fraction, y=CCpercInv]{../results/arenas-email/analysis_arenas-email.txt};
\addplot+ [smooth, semithick] table [x=Fraction, y=SCpercInv]{../results/arenas-email/analysis_arenas-email.txt};
\addplot+ [smooth, semithick] table [x=Fraction, y=BCpercInv]{../results/arenas-email/analysis_arenas-email.txt};
\end{axis}
\end{tikzpicture}

\bigskip

\begin{tikzpicture}
\begin{axis}
[
  height=0.3\textheight,
  width = 0.92\textwidth,
  xlabel={$p$},
  ylabel={Fraction of pairs},
  xlabel near ticks, ylabel near ticks,
  label style = {
    font = \footnotesize
  },
  ticklabel style = {
    font = \footnotesize
  },
  yticklabel style = {/pgf/number format/fixed},
  scaled y ticks = false,
  cycle list name=color list,
  legend style = {
    draw=none,
    font=\footnotesize
  },
  legend plot pos = left,
  legend cell align = left,
  legend entries = {$C_C$, $S_C$, $B_C$},
  title = {Inversion},
  title style = {
    font=\footnotesize
  }
]
\addplot+ [smooth, semithick] table [x=Fraction, y=CCpercInv]{../results/opsahl-powergrid/analysis_opsahl-powergrid.txt};
\addplot+ [smooth, semithick] table [x=Fraction, y=SCpercInv]{../results/opsahl-powergrid/analysis_opsahl-powergrid.txt};
\addplot+ [smooth, semithick] table [x=Fraction, y=BCpercInv]{../results/opsahl-powergrid/analysis_opsahl-powergrid.txt};
\end{axis}
\end{tikzpicture}
\caption{Pairs in wrong rank order with approximated centralities}
\label{fig:inversion}
\end{figure}
