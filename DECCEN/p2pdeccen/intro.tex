% !TeX root = p2p.tex

\theoremstyle{definition}
\newtheorem*{defcc}{Closeness centrality}
\newtheorem*{defsc}{Stress centrality}
\newtheorem*{defbc}{Betweenness centrality}

\section{Introduction}

The task of the project was to define \deccen{}, a decentralized algorithm able to compute the stress centrality values of a network (based on the ideas introduced in \cite{lehmann2003}), and study its behavior using the \emph{Peersim} simulator \cite{peersim}. Additionally -- given the high cost of the algorithm -- approximation techniques could be proposed in order to reduce the communication overhead and memory requirements, and the computation could be extended to also include closeness and betweenness centrality values.

This report defines \deccen{} extended to compute the three centrality indices mentioned. Furthermore, a different decentralized algorithm is defined, based on the previous works of Brandes, and Eppstein and Wang \cite{brandes2001, ew2004, brandes2007}, to approximate centrality indices by involving only a limited amount of agents in the computation. This second algorithm, labeled \multibfs{}, has a significantly smaller communication overhead than \deccen{}.

This document is organized as follows: in section 2 some preliminary definitions are given, while \deccen{} is defined in section 3 and \multibfs{} is defined in section 4. Experimental results are shown in section: the aims of the experiments were to compare the performance of the two algorithms, and to evaluate the quality of the estimates obtained with \multibfs{}. Follow a small overview of the most relevant choices made in the development of the project code, and the proofs of some results introduced in section 4.


\section{Preliminary definitions and assumptions}

The task is to compute centrality indices for a given undirected graph $G = (V,E)$, which is assumed to be connected. As usual, $n$ is the number of nodes $|V|$ and $m$ is the number of edges $|E|$. Each node $v \in V$ represents an independent agent with some given computational power, that can only communicate with its neighbors $N_v = \{u \in V : \{u,v\} \in E\}$. A path $p(s,t)$ from a source $s$ to a destination $t$ is a sequence of edges that connects the two endpoints. The distance $d(u,v)$ between two nodes is the length of the shortest path that connects them (with $d(u,u) = 0$) while the diameter $\Delta$ is the maximum distance between any pair of nodes. Note that $d(u,v) = d(v,u)$ since the graph is undirected. A node $v$ is a predecessor of $w$ with respect to a source $s$ if $\{v,w\} \in E$ and $d(s,v) +1 = d(s,w)$. The \emph{predecessor set} $P_s(w)$ of $w$ is the set of all predecessors of $w$ with respect to $s$.

The number of different shortest paths that connect two nodes $s,t \in V$ is denoted by $\sigma_{st}$, while the quantity $\sigma_{st}(v)$ is the number of shortest paths between $s$ and $t$ that pass through $v$ (this means that if $v = s$ or $v = t$ then $\sigma_{st}(v)$ is always zero).

The centrality indices relevant to this document are the following:
\begin{defcc} The \emph{closeness centrality} $C_C(v)$ of a node $v \in V$ is defined as
\begin{equation}
 C_C(v) = \frac{\sum_{u \in V} d(u,v)}{n-1}
\end{equation}
\end{defcc}
\begin{defsc} The \emph{stress centrality} $S_C(v)$ of a node $v \in V$ is defined as
\begin{equation}
 S_C(v) = \sum_{s \in V} \sum_{t \in V} \sigma_{st}(v)
\end{equation}
\end{defsc}
\begin{defbc} The \emph{betweenness} centrality $B_C(v)$ of a node $v \in V$ is defined as
\begin{equation}
 B_C(v) = \sum_{s \in V} \sum_{t \in V} \frac{\sigma_{st}(v)}{\sigma_{st}}
\end{equation}
\end{defbc}

The algorithms described in this report all assume an underlying synchronous model where the computation evolves in steps: at each step all the agents perform their computations independently and autonomously, and the messages they send at step $t$ are delivered to the destination and processed at step $t+1$.