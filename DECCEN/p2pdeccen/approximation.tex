% !TeX root = p2p.tex

\section{Approximation of centrality indices}

The main issue with the \deccen{} algorithm is its computational cost both in terms of memory consumption and number of messages exchanged, rendering the algorithm impractical for networks of reasonable size.. In order to keep track of the forwarded report messages and guarantee the termination of the protocol each node needs to maintain a data structure of size $O(n^2)$, while the number of messages exchanged is $O(n^2m)$.

However, if we are for example interested in computing centrality indices in order to mitigate network congestion a simple estimation of the values could be sufficient. In the following I propose an approximation algorithm that adapts an idea originally developed for closeness centrality in \cite{ew2004} and expanded for the estimation of betweenness centrality in \cite{brandes2007}. The idea is to isolate the contribution of a single node $s$ to the centrality values of all the other nodes in the network and compute those contributions by solving a Single-Source-Shortest-Path problem starting from $s$. We can then compute the centrality of a node $v$ by adding the contributions of all the other nodes to $v$ (that is, by solving $n$ SSSP instances starting from all the nodes and accumulating the contributions locally). In our case, the SSSP problem is converted to a decentralized Breadth First Search.

The approximation algorithms described in \cite{ew2004,brandes2007} estimate the centrality values by solving the SSSP problems for a restricted set of source nodes.

\subsection*{Contribution of a source to Closeness centrality}

The contribution of a source $s$ to the closeness centrality value of $v$ is simply the distance $d(s,v)$ of $v$ from $s$:

\begin{equation}
\gamma(s|v) = d(s,v) .
\end{equation}

\subsection*{Contribution of a source to Betweenness centrality}

The contribution of a source $s$ to the betweenness centrality of $v$ is the \emph{dependency} of $s$ on $v$ introduced in \cite{brandes2001}:

\begin{equation}
\delta(s|v) = \sum_{t \in V} \frac{\sigma_{st}(v)}{\sigma_{st}},
\end{equation}
that allows to rewrite the betweenness centrality of $v$ as

\begin{equation*}
BC(v) = \sum_{s \in V} \delta(s|v).
\end{equation*}

\subsection*{Contribution of a source to Stress centrality}

The contribution to stress centrality is analogous to the betweenness centrality:

\begin{equation} \label{eq:contrib:sc}
\sigma(s|v) = \sum_{t \in V} \sigma_{st}(v),
\end{equation}
and the stress centrality of $v$ is rewritten as

\begin{equation*}
SC(v) = \sum_{s \in V} \sigma(s|v).
\end{equation*}

\subsection{Computing contributions from a single source}
\label{sec:recursive}

As stated previously, a decentralized Breadth First Search can be adapted to compute the contribution of a source to any of the three centrality indices considered. The closeness centrality contribution is simply the distance from the source to the node (that is, the depth of the visited node in the Breadth-First tree) and it can be computed directly during the visit.

For betweenness centrality, \cite{brandes2001} shows that dependencies of a source obey a recursive relation expressed in terms of predecessors set:
\begin{theorem}[Brandes, 2001]
\label{th:contrib:bc}
The dependency of $s \in V$ on any $v \in V$ obeys
\begin{equation} \label{eq:th:contrib:bc}
\delta(s|v) = \sum_{w : v \in P_s(w)} \frac{\sigma_{sv}}{\sigma_{sw}} \cdot (1 + \delta(s|w)) .
\end{equation}
\end{theorem}
For stress centrality contributions a similar relation holds (the proof is reported in the appendix):
\begin{theorem}
\label{th:contrib:sc}
The stress centrality contribution of $s \in V$ on any $v \in V$ obeys
\begin{equation} \label{eq:th:contrib:sc}
\sigma(s|v) = \sum_{w : v \in P_s(w)} \sigma_{sv} \cdot \left( 1 + \frac{\sigma(s|w)}{\sigma_{sw}} \right) .
\end{equation}
\end{theorem}

The predecessor sets of all the nodes can be easily discovered by adjusting the ``descent'' phase of the BFS algorithm, while contributions are computed during a backward walk from the frontier of the BF-Tree back to the source.

\subsection{Random sampling of source nodes [FIXME]}

To let the algorithm operate in a decentralized way, each node independently initiates a visit with a given probability $p$, which reflects the fraction of the network sampled. Even if the number of samples is not known beforehand, eventually all the nodes will become aware of the number of sources that initiated a visit (let it be $k$).

The contributions of sample $v_i$ to the closeness, stress and betweenness centrality of any node $u$ can be modeled with the following random variables

To estimate the value of centrality indices, first the result of sampling from a source node $v_i$ at a node $u$ is modeled using the following random variables:
\begin{eqnarray*}
X_i(u) = \frac{n}{n-1} \cdot d(v_i,u) , \quad
Y_i(u) = n \cdot \delta(v_i|u) , \quad
Z_i(u) = n \cdot \sigma(v_i|u) ,
\end{eqnarray*}
then centrality indices are approximated with the following estimators:
\begin{eqnarray*}
\widehat{C}_C(u) = \sum_{i=1}^k \frac{X_i(u)}{k}, \quad
\widehat{C}_B(u) = \sum_{i=1}^k \frac{Y_i(u)}{k}, \quad
\widehat{C}_S(u) = \sum_{i=1}^k \frac{Z_i(u)}{k} .
\end{eqnarray*}

The derivations for the following theorem -- which ensures the estimators are unbiased -- are reported in the appendix.

\begin{theorem}[Unbiased estimators]
\label{th:expect}
The expected values of the centrality estimators are the exact centrality indices:
 \begin{enumerate}[label=\textup{(\alph*)}]
  \item $\mathbf{E}[\widehat{C}_C(u)] = C_C(u)$,
  \item $\mathbf{E}[\widehat{C}_S(u)] = C_S(u)$,
  \item $\mathbf{E}[\widehat{C}_B(u)] = C_B(u)$.
  \end{enumerate}
\end{theorem}

\subsection{Algorithm specification}

In this section the algorithm is detailed. The only parameter of the algorithm is the probability $p$ with which each independent node decides to begin a visit of the network. The network size $n$ is assumed to be known to all the agents in the network. Initially, every node sets to zero the estimation of each centrality index.

\subsubsection{Message types}

\paragraph{\mdiscargs{s}{u}{d}} Messages of this type are used during the descent from a source to build the predecessors sets at each node. Relevant fields are the source $s$ of the visit, the sender $u$, the distance $d$ of the sender to the source (that is, $d = d(s,u)$) and the number of shortest path from the source to the sender $\sigma_{su}$.

\paragraph{\mrepargs{s}{v}} These messages are sent by a node $v$ as part of the backtracking phase to inform its predecessors of the computed contributions and to allow them to compute their own by applying the recursive relations introduced in section \ref{sec:recursive}.

\subsubsection{Visit states}

Visit states are parametric with respect to the discovery from a source $s \in V$. A node $v$ is in state:
\begin{description}
\item[\swait{s}] if it has not yet received any \mdisc{} having $s$ as source.
\item[\sact{s}] if it has received one or more \mdisc{} messages with source $s$ and has not yet computed the contributions of $s$ to its centrality indices.
\item[\scomp{s}] if it has computed the contributions of $s$ to its centrality indices, updated them accordingly, and reported the contributions to each predecessor in $P_s(v)$ by sending the appropriate \mrep{} messages.
\end{description}

\subsubsection{Node state}
Each node $v$ maintains three centrality accumulators $C_C$, $C_B$ and $C_S$ like in \deccen{}, and a counter $k$ to track the number of sample nodes involved in the protocol.

Furthermore, while a node is in state \sact{s} when dealing with a visit from $s$ it will need to partition the set of neighbors $N_v$ in three subsets: the set of predecessors $P_s(v)$, the set of siblings $S_s(v)$ and the set of children $C_s(v)$, as well as track contributions of $s$ to its centrality scores with three parametric accumulators $C_C(s)$, $C_B(s)$ and $C_S(s)$.

\subsubsection{Protocol initialization}
Upon initialization, a node $v$ clears its centrality accumulators and the counter $k$, and enters state \swait{s} for all $s \in V$. Then, with probability $p$ initiates a visit by entering state \sact{v}, sending a \mdiscargsfull{v}{v}{0}{1} to every neighbor and letting $P_v(v) = \emptyset$.

\subsubsection{Step actions}
\label{multibfs:step}
The actions performed by a node $v$ at each step are the following:

\begin{algosteps}
  \item The messages received at the current step are divided by type and then \mdisc{} messages are grouped by source.
  \item For each group of \mdiscargs{s}{u}{d} messages having source $s$ and distance $d$ with $v$ in state \swait{s}:
  \begin{algosteps}
    \item Change state to \sact{s}, let $C_C(s) \gets d+1$, $C_B(s) \gets 0$, $C_S(s) \gets 0$, collect all the senders $u$ in the predecessor set $P_s(v)$. Let $\sigma_{sv} = \sum_{u \in P_s(v)} \sigma_{su}$.
    \item If $P_s(v) = N_v$ send a \mrepleaf{s}{v} message to all the predecessors $u \in P_s(v)$ and change state to \scomp{s}. 
    \item Otherwise, send to all $w \in N_v \setminus P_s(v)$ a \mdiscargs{s}{v}{d+1} message and change state to \sact{s}.
  \end{algosteps}
    \item For each group of \mdiscargs{s}{u}{d} messages having source $s$ and distance $d$ with $v$ in state \sact{s}:
    \begin{algosteps}
      \item Under the synchronous model assumption $v$ can only receive such messages from nodes $u$ such that $d(s,u) = d(s,v)$. Collect these nodes in the set $S_s(v)$ of the siblings of $v$ with respect to $s$. (Note that these messages are received exactly one step after $v$ has been contacted by its predecessors).
    \end{algosteps}
  \item For each \mrepargs{s}{w} message received with $v$ in state \sact{s}:
  \begin{algosteps}
      \item Add $w$ to the children set $C_s(v)$.
      \item Update $C_B(s) \gets C_B(s) + \frac{\sigma_{sv}}{\sigma_{sw}} \cdot (1 + \delta(s|w))$.
      \item Update $C_B(s) \gets C_B(s) + \sigma_{sw} \cdot ( 1 + \sigma(s|w)/\sigma_{sw} )$.
    \end{algosteps}
  \item For any $s \in V$ such that $v$ is in state \sact{s}, if $P_s(v) \cup S_s(v) \cup C_s(v) = N_v$ then:
  \begin{algosteps}
      \item If $s \neq v$ update the accumulators:
      \begin{itemize}
        \item[-] $C_C \gets C_C + C_C(s)$
        \item[-] $C_B \gets C_B + C_B(s)$
        \item[-] $C_S \gets C_S + C_S(s)$
      \end{itemize}
      and send a \mrepargs{s}{v} message to all the predecessors contained in $P_s(v)$.
      \item Increment the counter $k$ and change state to \scomp{s}.
    \end{algosteps}
\end{algosteps}
A node $v$ can obtain the value of the estimators in the following way:
\begin{eqnarray*}
\widehat{C}_C(v) =  \left( \frac{n}{k} \, C_C \right)^{-1}, \quad
\widehat{C}_B(v) = \frac{n}{k} \, C_B, \quad
\widehat{C}_S(v) = \frac{n}{k} \, C_S .
\end{eqnarray*}

Note that if the algorithm is executed with $p=1$ this is basically the decentralized version of the algorithm described in \cite{brandes2001}, and the estimators are actually the exact centrality indices.[FIXME move this at the beginning of the description]

\subsection{Cost analysis}

Each independent search requires $O(m)$ messages for the \mdisc{} phase and $O(m)$ messages to backtrack with \mrep{} messages. The parameter $p$ determines the fraction of nodes $pn$ that initiate a \mdisc{} search, so the total number of messages is $O(\lceil pn \rceil m)$.

In terms of memory consumption, note that for each $v \in V$ there are at most $\lceil pn \rceil$ other nodes $s$ for which $v$ is in state \sact{s} and needs to partition its neighbor set $N_v$ in the three subsets $P_s(v)$, $S_s(v)$ and $C_s(v)$, so the cost is $O(\lceil pn \rceil deg(v))$.