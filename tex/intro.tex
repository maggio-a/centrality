% !TeX root = p2p.tex

\theoremstyle{definition}
\newtheorem*{defcc}{Closeness centrality}
\newtheorem*{defsc}{Stress centrality}
\newtheorem*{defbc}{Betweenness centrality}

\section{Introduction}

The goal of the project was to define \deccen{}, a decentralized algorithm to compute stress centrality indices in a network based on the framework introduced in \citet{lehmann2003}, and study its behavior using the \peersim{} \cite{peersim} simulator. Additionally -- given the high cost of the algorithm -- approximation techniques could be proposed in order to reduce the communication overhead and memory requirements, and the algorithm could be extended to also compute closeness and betweenness indices.

This report defines a version of \deccen{} capable of computing all the three centrality measures mentioned. It also defines a different decentralized algorithm, based on the previous works of Brandes \cite{brandes2001,brandes2007} and Eppstein and Wang \cite{ew2004}, to approximate centralities by sampling the contributions of a subset of network nodes to their values. This second algorithm, referred to as \multibfs{}, has a significantly smaller communication overhead.

This document is organized as follows: in section \ref{sec:def} some preliminary definitions are given, while \deccen{} is defined in section \ref{sec:deccen} and \multibfs{} is defined in section \ref{sec:mbfs}. Section \ref{sec:exp} reports some experimental results: the aim of the experiments was to compare the performance of the two algorithms, and to evaluate the quality of the approximations obtained with \multibfs{}. Finally, section \ref{sec:code} contains a small overview of the most relevant choices made in the development of the simulator.


\section{Preliminary definitions and assumptions}
\label{sec:def}

The task is to compute centrality indices for a given undirected graph $G = (V,E)$, which is assumed to be connected. Unless otherwise stated, $n$ denotes the number of nodes $|V|$ and $m$ the number of edges $|E|$. Each vertex $v \in V$ represents a network node with some given computational power, that can only communicate with its direct neighbors $N_v = \{u \in V : \{u,v\} \in E\}$. The terms \emph{vertex} and \emph{node} will be used interchangeably.

A \emph{path} of \emph{length} $k$ from a source $s \in V$ to a destination $t \in V$ is a sequence $\langle v_0, v_1, \dots, v_k \rangle$ of vertices such that $s=v_0$, $t=v_k$ and $\{v_{i-1}, v_i\} \in E$ for $i = 1,2,\dots,k$. The \emph{distance} $d(u,v)$ between two vertices is the length of the shortest path that connects them (note that $d(u,v) = d(v,u)$ since the graph is undirected) and the diameter $\Delta$ is the maximum distance between any pair of vertices. A vertex $v$ is a \emph{predecessor} of $w$ with respect to $s$ if $\{v,w\} \in E$ and $d(s,v) +1 = d(s,w)$. The \emph{predecessor set} $P_s(w)$ of a vertex $w$ is the set of all predecessors of $w$ with respect to $s$.

The number of different shortest paths that connect two vertices $s,t \in V$ is denoted by $\sigma_{st}$, while the quantity $\sigma_{st}(v)$ is the number of shortest paths between $s$ and $t$ that pass through $v$ (this means that if $v = s$ or $v = t$ then $\sigma_{st}(v)$ is always zero).

The centrality indices relevant to this document are the following:
\begin{defcc} The \emph{closeness} centrality $C_C(v)$ of a vertex $v \in V$ is
\begin{equation}
 C_C(v) = \frac{\sum_{u \in V} d(v,u)}{n-1}
\end{equation}
\end{defcc}
\begin{defsc} The \emph{stress} centrality $S_C(v)$ of a vertex $v \in V$ is
\begin{equation}
 S_C(v) = \sum_{s \in V} \sum_{t \in V} \sigma_{st}(v)
\end{equation}
\end{defsc}
\begin{defbc} The \emph{betweenness} centrality $B_C(v)$ of a vertex $v \in V$ is
\begin{equation}
 B_C(v) = \sum_{s \in V} \sum_{t \in V} \frac{\sigma_{st}(v)}{\sigma_{st}}
\end{equation}
\end{defbc}

The algorithms described in this report all assume an underlying syn\-chro\-nous communication model where the computation evolves in steps: at each step all the network nodes perform their computations independently and autonomously, and the messages they send at step $t$ are delivered to the destinations and processed at step $t+1$.